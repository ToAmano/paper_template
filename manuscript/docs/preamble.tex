% preamble

% This is for template only !!
\usepackage{lipsum}

%\usepackage{floatrow}
\usepackage{siunitx}
\usepackage{color}
\usepackage{hhline}
\usepackage{mathrsfs}
\usepackage[dvipdfmx]{graphicx}
\usepackage{adjustbox}
\usepackage{dcolumn}
\usepackage{bm}% bold math
\usepackage{multirow}
\usepackage{booktabs}
\usepackage{afterpage}
\usepackage{amsmath}
\usepackage{ulem}
\usepackage{physics}

% https://mathlandscape.com/latex-underline/
\usepackage{ulem}

\usepackage{here} % force figures Here.

\usepackage[inline]{asymptote}

\usepackage[compat=1.1.0]{tikz-feynhand} % feynman diagram
% subcaption http://www.yamamo10.jp/yamamoto/comp/latex/make_doc/insert_fig/index.php#SUBCAP
% https://oku.edu.mie-u.ac.jp/tex/mod/forum/discuss.php?d=1024
% https://atatat.hatenablog.com/entry/cloud_latex18_subcaption
\usepackage{caption}
\usepackage[subrefformat=parens]{subcaption}
%\usepackage{floatrow}
%\usepackage[export]{adjustbox}

% http://www.yamamo10.jp/yamamoto/comp/latex/make_doc/chemistry/index.php
\usepackage[version=4]{mhchem}

% for tikz 
\usepackage{tikz, pgf, pgfplots, pgfplotstable}
\usepackage{tikz-3dplot}
\usetikzlibrary{math,calc}
\pgfplotsset{compat = newest}
\usepgfplotslibrary{groupplots} % LATEX and plain TEX
\pgfplotsset{compat = newest}

\pgfplotsset{
   table/search path={figures/self_energy, figures/tdos, figures/diel_func, figures/band},
}

\graphicspath{{figures/self_energy, figures/tdos, figures/diel_func, figures/band}}

% https://tikz.dev/library-external
% crisさんおすすめ
% \usetikzlibrary{external}
% \tikzexternalize % activate!

\usepackage{multirow}

\usepackage[nohyperlinks,nolist]{acronym} % abbreviation

\usepackage[subpreambles=true,sort=true]{standalone}

%https://uec.medit.link/latex/table.html#:~:text=%E3%82%BB%E3%83%AB%E3%81%AE%E7%B5%90%E5%90%88,%E3%82%92%E8%AA%AD%E3%81%BF%E8%BE%BC%E3%82%80%E5%BF%85%E8%A6%81%E3%81%8C%E3%81%82%E3%82%8B%E3%80%82

% \usepackage[no-math,haranoaji,deluxe]{luatexja-preset}
%https://mizunashi-mana.github.io/blog/posts/2021/12/migrate-to-luatexja/

\newcommand{\red}[1]{\textcolor{red}{#1}}
\newcommand{\blue}[1]{\textcolor{blue}{#1}}
\arraycolsep=0.0em
\setlength{\abovecaptionskip}{0mm}
\setlength{\belowcaptionskip}{0mm}
\usepackage{caption} 
\captionsetup[table]{skip=8pt}
\captionsetup[figure]{skip=4pt}
%\setlength{\MidlineHeight}{2pt}


\usepackage{comment}
%\usepackage{natbib}

\usepackage[colorlinks=true,citecolor=blue,linkcolor=blue,urlcolor=blue]{hyperref}
\usepackage{cleveref}
\crefname{equation}{Eq.}{Eq.}% {環境名}{単数形}{複数形} \crefで引くときの表示
\crefname{figure}{Fig.}{Fig.}% {環境名}{単数形}{複数形} \crefで引くときの表示
\crefname{table}{Table}{Table}% {環境名}{単数形}{複数形} \crefで引くときの表示
\crefname{section}{Sec.}{Sec.}% {環境名}{単数形}{複数形} \crefで引くときの表示
\crefname{appendix}{Appendix}{Appendix}% {環境名}{単数形}{複数形} \Crefで引くときの表示


\Crefname{equation}{Equation}{Equation}% {環境名}{単数形}{複数形} \Crefで引くときの表示
\Crefname{figure}{Figure}{Figure}% {環境名}{単数形}{複数形} \Crefで引くときの表示
\Crefname{table}{Table}{Table}% {環境名}{単数形}{複数形} \Crefで引くときの表示
\Crefname{section}{Section}{Section}% {環境名}{単数形}{複数形} \Crefで引くときの表示
\Crefname{appendix}{Appendix}{Appendix}% {環境名}{単数形}{複数形} \Crefで引くときの表示



\usepackage{threeparttable} %https://qiita.com/kumamupooh/items/38795811fc6b934a950d
% bookmark
% \usepackage{xurl}
% \hypersetup{unicode,bookmarksnumbered=true,hidelinks,final}


\newcommand{\ph}{\phantom{0}}

\renewcommand{\topfraction}{1.0}
\renewcommand{\bottomfraction}{1.0}
\renewcommand{\dbltopfraction}{1.0}
\renewcommand{\textfraction}{0.1}
\renewcommand{\floatpagefraction}{0.9}
\renewcommand{\dblfloatpagefraction}{0.9}

