%
% 2023/11/1
%
%
%

\documentclass[tikz]{standalone}
% https://latexdraw.com/plot-a-function-and-data-in-latex/

\usepackage{tikz}
\usepackage{pgfplots}
\usepackage{physics}

\pgfplotsset{compat = newest}
\pgfplotsset{every axis/.append style={
                    label style={font=\Large},
                    tick label style={font=\Large},
                   }}
\begin{document}

\def\eps0{300}
\def\epsinf{100}
\def\tau{0.01}
% \def\freq0{500};


    \begin{tikzpicture}
     \tikzset{every node}=[font=\Large]
        % axis環境が2次元plot
     \begin{axis}[ %グラフ設定
	    xmode=log,
            xmin = 0, xmax = 100/\tau,
            % ymin = -0.003, ymax = 0.010,
            ticks=none,
	    xlabel=$\log \left(\mathrm{Frequency}\right)$,
	    ylabel=$\varepsilon(\omega)$,
            % grid = both,
            minor tick num = 1,
            major grid style = {lightgray},
            minor grid style = {lightgray!25},
            width = 8.6cm, % \textwidth,
            % height = 0.75\textwidth,
            legend cell align = {left},
            legend pos = north east
        ]
            %\coordinate (A) at (80,-1); 

	    %\coordinate (B) at (80,5);
	 
	 
	 % 実部
	 \addplot[domain = 0.01/\tau:100/\tau,samples=1000,thick,blue] {\epsinf+(\eps0-\epsinf)/(1+\tau*\tau*x*x)};
	 % 虚部
	 \addplot[domain = 0.01/\tau:100/\tau,samples=1000,thick,red] {(\eps0-\epsinf)*\tau*x/(1+\tau*\tau*x*x)};


%            \addplot[blue,solid, line width = 2, mark = none] table [x index=2, y index=4] {\au};
%            \addplot[red ,solid, line width = 2, mark = none] table [x index=2, y index=4] {\eu};
	    % \addplot[blue, dotted, line width = 2, mark=none] coordinates {(80,-1) (80,5)}; 
	    % \addplot[orange, dotted, line width = 2, mark=none] coordinates {(\freq0,-0.002) (\freq0,0.01)};  % !! これはつかっても良いもか
%	    \addplot[blue,dotted, line width = 2, mark=none] coordinates {(598,0) (598,180)};
            %\addplot[red, only marks] table [x ={x}, y = {y2}] {\table};
            %\addplot[teal, only marks, mark = x, mark size = 3pt]
            %    table [x = {x}, y = {y3}] {\table};
	    % \node[] at (\freq0,-0.001) {$\omega_0$}; % !! これは使ってもいいかも.
	    % \node[] at (170,-0.5) {$\omega_0+\Delta\omega$};
	    % \node[] at (90,2) {$\gamma$};
            \legend{
	       $\Re\varepsilon(\omega)$,
	       $\Im\varepsilon(\omega)$} 
	    %,
            %    Plot only with marks,
            %    Plot with other type of marks}
     \end{axis}
    \end{tikzpicture}
\end{document}

% \documentclass{standalone}

% \usepackage{tikz}
% \usepackage{pgfplots}

% \pgfplotsset{compat = newest}

% \begin{document}
%     \begin{tikzpicture}
%         \begin{axis}[
%             xmin = 0, xmax = 30,
%             ymin = -1.5, ymax = 2.0,
%             xtick distance = 2.5,
%             ytick distance = 0.5,
%             grid = both,
%             minor tick num = 1,
%             major grid style = {lightgray},
%             minor grid style = {lightgray!25},
%             width = \textwidth,
%             height = 0.5\textwidth,
%             xlabel = {$x$},
%             ylabel = {$y$},
%             legend cell align = {left},
%         ]
%             \addplot[
%                 domain = 0:30,
%                 samples = 200,
%                 smooth,
%                 thick,
%                 blue,
%             ] {exp(-x/10)*( cos(deg(x)) + sin(deg(x))/10 )};
            
%             \addplot[
%                 smooth,
%                 thin,
%                 red,
%                 dashed
%             ] file[skip first] {cosine.dat};
            
%             \legend{Plot from expression, Plot from file}
%         \end{axis}
%     \end{tikzpicture}
% \end{document}