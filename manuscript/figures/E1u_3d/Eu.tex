\documentclass[lualatex]{standalone}
\usepackage{asymptote}

\begin{document}
\begin{asy}
import three;
settings.render=8;
settings.prc=false;
settings.outformat="pdf";
// unitsize(1cm);
size(4cm,3cm);

// the camera position 
currentprojection = orthographic((3,1,0.5)); 

// 
material sphereCcolor = material(diffusepen=gray, specularpen=white); //material1
material O_color = material(diffusepen=red, specularpen=white); //material1
material cylcolor = material(diffusepen=gray, emissivepen=gray);//material2

// cylinder raduis
real cylRadius = 0.1;
// point radius
real sphereRadius = 0.3;

// draw rod(line)
void drawRod(triple a, triple b) {
  surface rod = extrude(scale(cylRadius)*unitcircle, axis=length(b-a)*Z);
  triple orthovector = cross(Z, b-a);
  if (length(orthovector) > .01) {
    real angle = aCos(dot(Z, b-a) / length(b-a));
    rod = rotate(angle, orthovector) * rod;
  }
  draw(shift(a)*rod, surfacepen=cylcolor);
}

// draw carbon(point)
void drawCarbon(triple center) {
     draw(shift(center)*scale3(sphereRadius)*unitsphere, surfacepen=sphereCcolor);
}


// draw carbon(point)
void drawOCarbon(triple center) {
     draw(shift(center)*scale3(sphereRadius)*unitsphere, surfacepen=O_color);
}

// draw arrow
void drawArray(triple center, triple direction) {
     triple end = center+direction;
     draw(center--end, green, Arrow3);
}


// variable
real u = 0.304*4;
real c = 0.644*4;

// variable
triple dir = (0,0.8, 0);


// triple : an ordered triple of real components (x,y,z)
// corner and center (Ti)
triple Ti_a = (0,0,0);
triple Ti_b = (4,0,0);
triple Ti_c = (0,4,0);
triple Ti_d = (4,4,0);
triple Ti_e = (0,0,c);
triple Ti_f = (4,0,c);
triple Ti_g = (0,4,c);
triple Ti_h = (4,4,c);
triple Ti_i = (2,2,c/2);
// O
triple O_a  = (2-u,2+u,c/2);
triple O_b  = (2+u,2-u,c/2);
triple O_c  = (u,u,0);
triple O_d  = (4-u,4-u,0);
triple O_e  = (u,u,c);
triple O_f  = (4-u,4-u,c);



// lines 
drawRod(Ti_a,O_c);
drawRod(Ti_d,O_d);
drawRod(Ti_c,O_a);
drawRod(Ti_b,O_b);

drawRod(Ti_e,O_e);
drawRod(Ti_h,O_f);
drawRod(Ti_g,O_a);
drawRod(Ti_f,O_b);



drawRod(Ti_i,O_a);
drawRod(Ti_i,O_b);
drawRod(Ti_i,O_c);
drawRod(Ti_i,O_d);
drawRod(Ti_i,O_e);
drawRod(Ti_i,O_f);



// points
drawCarbon(Ti_a);
drawCarbon(Ti_b);
drawCarbon(Ti_c);
drawCarbon(Ti_d);
drawCarbon(Ti_e);
drawCarbon(Ti_f);
drawCarbon(Ti_g);
drawCarbon(Ti_h);
drawCarbon(Ti_i);
drawOCarbon(O_a);
drawOCarbon(O_b);
drawOCarbon(O_c);
drawOCarbon(O_d);
drawOCarbon(O_e);
drawOCarbon(O_f);


// arrows
drawArray(Ti_a,dir);
drawArray(Ti_b,dir);
drawArray(Ti_c,dir);
drawArray(Ti_d,dir);
drawArray(Ti_e,dir);
drawArray(Ti_f,dir);
drawArray(Ti_g,dir);
drawArray(Ti_h,dir);
drawArray(Ti_i,dir);
drawArray(O_a,-dir);
drawArray(O_b,-dir);
drawArray(O_c,-dir);
drawArray(O_d,-dir);
drawArray(O_e,-dir);
drawArray(O_f,-dir);



// Frame
material framecolor = material(diffusepen=black); // material3
void drawFrame(triple a, triple b) {
  surface rod = extrude(scale(.1*cylRadius)*unitcircle, axis=length(b-a)*Z);
  triple orthovector = cross(Z, b-a);
  if (length(orthovector) > .01) {
    real angle = aCos(dot(Z, b-a) / length(b-a));
    rod = rotate(angle, orthovector) * rod;
  }
  draw(shift(a)*rod, surfacepen=framecolor);
  draw(shift(b)*scale3(cylRadius)*unitsphere, surfacepen=framecolor);
}
drawFrame(Ti_a,Ti_b);
drawFrame(Ti_a,Ti_c);
drawFrame(Ti_b,Ti_d);
drawFrame(Ti_c,Ti_d);

drawFrame(Ti_a,Ti_e);
drawFrame(Ti_b,Ti_f);
drawFrame(Ti_c,Ti_g);
drawFrame(Ti_d,Ti_h);


drawFrame(Ti_e,Ti_f);
drawFrame(Ti_e,Ti_g);
drawFrame(Ti_f,Ti_h);
drawFrame(Ti_g,Ti_h);



\end{asy}
\end{document}
